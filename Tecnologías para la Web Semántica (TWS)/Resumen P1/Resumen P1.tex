\documentclass[a4paper,12pt,twoside,final,spanish]{article}
%titlepage: pone el título en una página aparte
%twocolumn
\usepackage{babel} %Para el lenguaje [spanish]
\usepackage[utf8]{inputenc} %Para reconocer todos los símbolos
\usepackage[T1]{fontenc}
\usepackage{textcomp}
\usepackage{amsmath}
\usepackage{amsfonts}
\usepackage{amssymb}
\usepackage[margin=2cm]{geometry} %Márgenes
\usepackage[T1]{fontenc}
\usepackage{graphicx}
\usepackage{enumerate}
%\usepackage{hyperref}
%\pagestyle{headings}
\usepackage{titletoc} %Para hacer el índice

%---
\usepackage{geometry} %Algo de las líneas del pie y encabezados
\geometry{text={7in,9.5in},headheight=15pt}
%\textwidth = 7 in
%\textheight = 9.5 in
%\oddsidemargin = -0.25 in
%\evensidemargin = 0.0 in
%\topmargin = -0.25 in
%\headheight = 0.0 in
%\headsep = 0.0 in
\setlength{\parskip}{0.1in}
\setlength{\parindent}{0.0in}
%---
\usepackage{fancyhdr} %Para usar encabezados y pies personalizados
	\pagestyle{fancy}
	\fancyhf{}
	% E: par
	% O: impar
	\fancyhead[LE,RO]{Tecnologías para la Web Semántica} 
	\fancyhead[RE,LO]{Resumen P1}
	\fancyfoot[RE,LO]{Darién Julián Ramírez}
	\fancyfoot[LE,RO]{\thepage}
	\renewcommand{\footrulewidth}{1pt}
%---
\usepackage{listings} %Para escribir códigos
\lstset{language=XML,
	basicstyle=\footnotesize,
	numbers=left,
 	stepnumber=1,
	numbersep=8pt,
	showspaces=false,               % show spaces adding particular underscores
  	showstringspaces=false,         % underline spaces within strings
  	frame=lines,                   % adds a frame around the code
	tabsize=4,                      
  	captionpos=b,                   % sets the caption-position to bottom
  	breaklines=true,                % sets automatic line breaking
}
%---

\title{\Huge Tecnologías para la Web Semántica\\
Resumen\\
Parcial 1}
\author{Darién Julián Ramírez}
\date{\vspace{-5ex}}

\begin{document}

\maketitle %Crea la página de título

\tableofcontents %Genera el índice

\cleardoublepage

\section{Visión de la Web Semántica}

\subsection{¿Cómo funciona la Web hoy?}

\begin{itemize}
\item Tráfico desde buscadores:
	\begin{itemize}
	\item Miles de resultados con poca precisión.
	\item Baja respuesta.
	\item Resultados altamente sensibles al vocabulario.
	\item  Resultados inconexos, páginas en vez de sitios.
	\end{itemize}
	
\item Recuperación de información:
	\begin{itemize}
	\item Intensiva en tiempo y trabajo del usuario.
	\item Información no clasificada.
	\item Informacion difícil de procesar.
	\end{itemize}
	
\item Aplicaciones aisladas.

\item Buscadores de sitios.

\item Ambigüedad (diferentes maneras de referirse a lo mismo) y falta de precisión (referencia a distintas entidades con el mismo término).

\item Los datos no están inmediatamente disponibles para ser procesados por otro software.

\item Los datos no están interconectados, cada sitio es una \textit{isla}.

\item No existe la capacidad de interpretar sentencias para extraer información útil.

\item La mayor parte del contenido Web está diseñado para la lectura de humanos, no para que los programas puedan manipularlos significativamente.

\item Un espacio de información global.

\item Recursos enlazados a otros recursos.

\item Representa la información usando lenguaje natural, gráficos, multimedia, diseños de las páginas.

\item Los humanos pueden procesar esta información fácilmente, deducen hechos desde información parcial, crean asociaciones mentales y asimilan información desde distintos sentidos.
\end{itemize}

\subsection{De la Web actual a la Web Semántica}

\begin{itemize}
\item Hacer que las máquinas entiendan significados.

\item Hacer los datos más inteligentes (W3C).

\item El objetivo es que la semántica se convierta en la protagonista.

\item La semántica es la parte de la lingüística que estudia la forma de las estructuras léxicas y los procesos mentales a través de los cuales los seres humanos damos sentido a las expresiones lingüísticas.

\item Dotando de más semántica a la web, lo que se busca es resolver los problemas que en la actualidad causan los entornos digitales carentes de semántica, dificultando en ocasiones la búsqueda de información.

\item Máquinas que comprendan significados.

\item Interoperabilidad de la información.

\item Búsquedas más eficientes: resultados precisos en menos tiempo.

\item Usuario ocupado en la toma de decisiones y no en las tareas repetitivas.
\end{itemize}

\subsection{¿Qué es la Web Semántica?}

\begin{itemize}
\item Una red de significados.

\item Información clasificada.

\item Jerarquía de datos.

\item Una web semántica es una red de datos que pueden ser procesados directa o indirectamente por máquinas. Es una web extendida que permitirá a humanos y máquinas trabajar en cooperación mutua.

\item Datos procesables por los humanos + \textbf{datos procesables por las máquinas}.

\item Desarrollo de lenguajes para expresar meta-información comprensible por maquinas.

\item Desarrollo de herramientas y arquitecturas que utilicen esos lenguajes.

\item Desarrollo de aplicaciones que provean un nuevo nivel de servicios.

\item La Web Semántica es una visión: La idea de tener datos en la Web definidos y vinculados de manera que puedan ser utilizados por máquinas no sólo con propósitos de muestra sino para automatización, integración y reutilización de datos entre diferentes aplicaciones.

\item Persigue el establecimiento de una forma universal de representar las relaciones  entre los datos y entre éstos y sus significados.

\item La promesa es una enorme facilidad para encontrar información relevante de forma  potencialmente sencilla.

\item Uso de metadatos, estándares, ontologías, lógica, motores de inferencia y agentes inteligentes.
\end{itemize}

\textbf{Semántica}: entendida como, significado procesable por máquinas.

\textbf{Metadatos}: entendidos como contenedores de información semántica sobre los datos.

\textbf{Estándares}: entendidos como, especificaciones que regulan la realización de ciertos procesos para garantizar la interoperabilidad.

\textbf{Ontologías}: entendidas como, el conjunto de términos y relaciones entre ellos que describen un dominio de aplicación concreto.

\textbf{Agente inteligente:} Un agente inteligente, es una entidad capaz de percibir su entorno, procesar tales percepciones y responder o actuar en su entorno de manera racional, es decir, de manera correcta y tendiendo a maximizar un resultado esperado.

\textbf{Buscador semántico:}  es aquél que realiza el rastreo atendiendo al significado del grupo de palabras que se escriben y no basándose en las actuales etiquetas.
En pocas palabras, un buscador inteligente.

\section{Anotaciones Semánticas I - Metadatos}

\subsection{Continuidad Semántica}

\subsubsection{Semántica implícita}
	\begin{itemize}
	\item Entendimiento compartido derivado del consenso humano.
	\item Desventajas:
		\begin{itemize}
		\item Ambigüedad.
		\item Desacuerdos en el significado de un término.
		\end{itemize}
	\end{itemize}
	
\subsubsection{Semántica informal (explícita)}
	\begin{itemize}
	\item Semántica explícita y expresada de manera informal.
	\item Especificadas por humanos en lenguaje de trabajo.
	\item Especificaciones UML utilizadas para el desarrollo de herramientas.
	\item Desventajas:
		\begin{itemize}
		\item Persistencia de ambigüedad.
		\item Baja probabilidad que dos implementaciones sean consistentes y 					compatibles.
		\item Problemas cuando se requiere interoperabilidad.
		\end{itemize}
	\item Esfuerzos para crear semánticas formales.
	\end{itemize}
	
\subsubsection{Semántica formal para procesamiento humano}
	\begin{itemize}
	\item Semántica explícita expresada en lenguaje formal sólo para procesamiento 			humano. Documentación formal.
	\item Axiomas y definiciones en Ontologías para empresas se crearon sin la 				expectativa de ser utilizadas para inferencia automatizada. El objetivo primario 		era ayudar a comunicar el significado pretendido a las personas.
	\item Desventaja: elimina la ambigüedad pero persiste la presencia de 					humanos en el ciclo.
	\end{itemize}

\subsubsection{Semántica Procesable por Máquinas}
	\begin{itemize}
	\item Representar las semánticas formalmente y permitir a las máquinas procesarlas 	para dinámicamente descubrir qué significa el contenido y cómo se utiliza.
	\end{itemize}


\subsection{Web Semántica}

¿Cómo puede una máquina (agente de software) aprender algo acerca del significado de un término que nunca antes había encontrado?

\begin{itemize}

\item Esta evolución tomorá lugar:
	\begin{itemize}
	\item Trasladándose a través de la contunuidad semántica desde semántica implícita 	a  semánticas formales procesables por máquinas.
	\item Reduciendo el contenido Web semántico no especificado.
	\item Desarrollando mapeos semánticos y capacidades de traducción donde las 			diferencias persistan.
	\end{itemize}

\item Frente a la semántica implícita, el crecimiento caótico de recursos, y la ausencia de una organización clara de la Web actual, la Web semántica aboga por clasificar, dotar de estructura y anotar los recursos con semántica explícita procesable por máquinas.

\item Colocar datos en contexto mediante la adición de metadatos.
\end{itemize}

\subsection{Metadatos}

\begin{itemize}
\item Los metadatos son datos acerca de los datos.

\item Datos acerca de los datos producidos por una institución y/o personas y de los servicios por ellas ofrecidos, proporcionados en forma estandarizada.

\item Información estructurada que se crea específicamente para describir recursos.

\item En diversos campos de la informática, como la recuperación de información o la Web Semántica, los metadatos en etiquetas son un enfoque importante para construir un puente sobre el intervalo semántico.

\item Mecanismo para etiquetar, catalogar, describir y clasificar los recursos presentes en la World Wide Web con el fin de facilitar la posterior búsqueda y recuperación de la información.

\item Dato estructurado sobre la información, o sea, información sobre información, datos sobre datos. 

\item Datos que se pueden guardar, intercambiar y procesar estructurados de tal forma que permiten ayudar a la identificación, descripción, clasificación y localización del 
contenido de un documento o recurso web y que sirven para su recuperación.

\item Tienen función de localización, identificación y descripción de recursos, legibles e interpretables por máquina.
\end{itemize}

\subsection{¿Qué es un estándar?}

Es una especificación que regula la realización de ciertos procesos o la fabricación de componentes para garantizar la interoperabilidad. Ejemplo: DublinCore.

\section{Anotaciones Semánticas II - Ontologías}

\subsection{¿Qué es una Ontología?}

\begin{itemize}
\item Define los términos y relaciones básicos que comprenden el vocabulario de un área así como las reglas que combinan términos y relaciones para definir extensiones del vocabulario.

\item Es una especificación explícita de una conceptualización.

\item Es una especificación formal de una conceptualización compartida

\item Puede tomar una variedad de formas, pero será necesario incluir un vocabulario de términos y algunas especificaciones sobre su significado. Esto incluye definiciones y una indicación de cómo se interrelacionan los conceptos, lo que colectivamente impone una estructura  en el dominio y restringe la posible interpretación de los términos.

\item Reutilizar: construir nuevas aplicaciones a partir de componentes existentes.

\item Compartir: utilización del mismo componente por diferentes recursos.

\item Usualmente se construyen en forma cooperativa por diferentes grupos de personas en diferentes ubicaciones.

\item Una ontología es una 6-tupla que consiste en conceptos, relaciones, jerarquías, una función que relaciona conceptos no-taxonómicamente, un conjunto de axiomas, y un conjunto de reglas.
\end{itemize}

\subsection{¿Para qué desarrollar una Ontología?}

\begin{itemize}
\item Para compartir entendimiento común de la estructura de la información entre personas o agentes de software.

\item Para permitir la reutilización del conocimiento de un dominio.

\item Para hacer explícitas las afirmaciones de un dominio.

\item Para separar el dominio del conocimiento del dominio operacional.

\item Para analizar el dominio del conocimiento.

\item Apuntan a capturar conocimiento consensuado de un modo genérico, para que este pueda ser reutilizado y compartido a través de aplicaciones de software y por grupos de personas.
\end{itemize}

\subsection{Tipos de Ontologías}

\subsubsection{Según la riqueza de su estructura interna}

\begin{itemize}
\item \textbf{Catálogo:}

Vocabularios controlados: lista finita de términos.

\begin{itemize}
\item Lista ordenada o clasificada sobre cualquier tipo de objetos (monedas, bienes a la venta, documentos, entre otros) o en su defecto personas.

\item Conjunto de publicaciones u objetos que se encuentran clasificados normalmente para la venta.

\item Publicación empresarial cuyo fin primero es el de la promoción de aquellos productos o servicios que una empresa ofrece.

\item Está compuesto principalmente por imágenes de los productos o servicios que se ofrecen en la empresa y que pueden ir acompañadas de breves descripciones, precio o beneficios del producto.

\item Es una comunicación visual de lo que se produce.

\item Pueden utilizarse para presentaciones individuales y detalladas de un determinado producto que por ejemplo está recién saliendo a la venta.
\end{itemize}

\item \textbf{Glosario:}

Son listas de términos con sus significados expresados en lenguaje natural. 

\begin{itemize}
\item Un glosario es un catálogo que contiene palabras pertenecientes a una misma disciplina o campo de estudio, apareciendo las mismas explicadas, definidas y comentadas.

\item Puede ser un catálogo de palabras desusadas o del conjunto de comentarios y glosas sobre los textos de un autor determinado.

\item Suele ser incluido al final o al comienzo de un libro o de una enciclopedia, con el objetivo de complementar la información que el mismo proporciona.

\item Son elaborados por especialistas en los campos sobre los cuales se ocupan y apuntan a llegar más allá de aquellos interesados en la materia sobre la cual se ocupan.

\item Ejemplos de glosarios:
	\begin{itemize}
	\item Glosario educativo.
	\item Glosario de términos.
	\item Glosario ambiental.
	\item Glosario informático.
	\item Glosario de salud.
	\item Glosario de términos médicos.	
	\end{itemize}
\end{itemize}

\item \textbf{Tesauro:}

Proveen semánticas adicionales entre términos, por ejemplo, información referida a sinónimos.

\begin{itemize}
\item Vocabulario controlado para representar de manera unívoca el contenido de los documentos y de las preguntas, así como para ayudar al usuario en el tratamiento de la información.

\item Diccionario que muestra la equivalencia entre los términos o expresiones del lenguaje natural y aquellos términos normalizados procedentes del lenguaje documental, así como las relaciones semánticas que existen entre los términos.

\item Compilación de palabras y frases que muestran sus sinónimos, sus jerarquías y cuya función es suministrar un vocabulario normalizado para la recuperación y almacenamiento de la información.
\item Norma ISO:
	\begin{itemize}
	\item Según su función un tesauro es un instrumento de control de la terminología 		que se utiliza mediante la transposición del lenguaje natural (utilizado por los 		usuarios, indexadores y en los documentos) a un lenguaje más estricto como es el 		documental. 
	\item Según su estructura es un vocabulario controlado y dinámico de términos con 		relaciones semánticas entre ellos y que se aplican a campos temáticos particulares 	del conocimiento.
	\end{itemize}

\item En líneas generales, un tesauro comprende lo siguiente:
	\begin{itemize}
	\item Un listado de términos preferidos, que se los ordena en forma alfabética, 
	temática y jerárquicamente.	
	\item Un listado de sinónimos de esos términos preferidos, llamados descriptores, 		con la leyenda "úsese (término preferido)" o una indicación similar.
	\item Una jerarquía o relaciones entre los términos. Esto se expresa con la 			identificación de \textit{términos más generales} y \textit{términos más 				restringidos}.
	\item Las definiciones de los términos, para facilitar la selección de los mismos 		por parte del usuario.
	\item Un conjunto de reglas para usar el tesauro.
	\end{itemize}
\end{itemize}

\item \textbf{\textit{Es un} - informal:}

Las jerarquías informales \textit{Es un}, son jerarquías de términos que no corresponden a una subclase estricta, por ejemplo, los términos \textbf{auto} de \textbf{alquiler} y \textbf{hotel} podrían ser modelados informalmente bajo la jerarquía \textbf{viaje} ya que se considerarían partes clave de un viaje.

Si \textbf{perfume} es subclase de \textbf{ropa}, heredaría la propiedad \textit{estaHechoDe} y heredaría los valores de dicha propiedad.

\item \textbf{\textit{Es un} - estricta:}

En este caso existe una relación estricta entre instancias de una clase y de las superclases correspondientes. Su objetivo es explotar el concepto de herencia.

Si B es subclase de A, entonces si un objeto es instancia de B, también es instancia de A. Las relaciones \textit{Es un} estrictas son necesarias para la explotación de la herencia.

\item \textbf{Marcos:}

Son ontologías que incluyen tanto clases como sus propiedades, las cuales pueden ser heredadas por otras clases en los niveles mas bajos de una taxonomía formal \textit{Es un}.

\item \textbf{Restricciones de valor:}

Expresan restricciones de acuerdo al tipo de dato de una propiedad, por ejemplo, tipo: fecha.

Las clases incluyen información acerca de propiedades. Se determinan restricciones en los valores de una propiedad. Ejemplo: La propiedad precio puede estar restringida entre ciertos valores (rango) y \textit{estaHechoDe} puede completarse seleccionando una lista de materiales.

\item \textbf{Restricciones lógicas:}

Expresan relaciones lógicas entre términos y otras relaciones mas detalladas, por ejemplo, la disyunción, relaciones inversas, etc. Son las ontologías más expresivas.
\end{itemize}

\subsubsection{Según su profundidad de modelado}

\begin{itemize}
\item \textbf{Ontologías livianas:}
	\begin{itemize}
	\item  Conceptos.
	\item Taxonomías de conceptos.
	\item Relaciones entre conceptos.
	\item Propiedades que describen relaciones.
	\end{itemize}

\item \textbf{Ontologías pesadas:}
	\begin{itemize}
	\item Modelan un dominio de manera más profunda.
	\item Agregan axiomas y restricciones a las anteriores.
	\end{itemize}
\end{itemize}

\subsection{Componentes}

Una ontología tiene los siguientes tipos de entidades:

\subsubsection{Clases o términos}

	\begin{itemize}
	\item Conjunto de individuos que tienen una característica común.
	\item Representa conceptos en un sentido amplio.
	\item Se organizan en taxonomías y se aplican mecanismos de herencia.
	\item Pueden representar conceptos abstractos o específicos.
	\end{itemize}

\textbf{Taxonomía:} vocabulario controlado ordenado jerárquicamente. Una taxonomía define la clasificación de términos y los principios que rigen esa clasificación. Las relaciones de una taxonomía están dadas por \textit{Es un}.

\textbf{Vocabulario controlado:} lista cerrada de términos definidos y únicos (sin ambigüedad).

\subsubsection{Propiedades}

	\begin{itemize}
	\item Enlazan individuos en parejas.
	\item Atributos que describen un objeto.
	\item Representa un tipo de asociación entre conceptos de un dominio 					(interacción).
	\end{itemize}
	
\subsubsection{Individuos o instancias}

	\begin{itemize}
	\item Los objetos del dominio a representar.
	\item Representa miembros de una clase o concepto (objetos) indivisibles.
	\end{itemize}
	
\subsubsection{Axiomas}

	\begin{itemize}
	\item Modelan sentencias que son siempre verdaderas.
	\item Representan conocimiento que no puede ser formalmente definido por los otros 	componentes.
	\item Se utilizan para verificar la consistencia del conocimiento almacenada en 		una base de conocimiento
	\item Permiten inferir nuevo conocimiento.
	\end{itemize}

\subsection{Clasificación}

\subsubsection{Ontologías de Alto Nivel}

\begin{itemize}
\item Describen y proponen conceptos generales a los que todos lo términos en ontologías existentes deberían vincularse. Ejemplo: espacio, tiempo, materia, objeto.
\item Los términos son los mismos a través de diferentes dominios de conocimiento.
\item Son independientes de un dominio o problema particular.
\item Su intención es unificar criterios entre grandes comunidades de usuarios. 
\end{itemize}

\subsubsection{Ontologías de Dominio}

\begin{itemize}
\item Describen el vocabulario relacionado a un dominio genérico, por ejemplo,  medicina, por medio de la especialización de los conceptos introducidos en las  ontologías de alto nivel.
\item Son reutilizables en un dominio especifico dado.
\item Proveen vocabulario acerca de conceptos dentro de un dominio y sus relaciones, a las actividades que se realizan dentro del dominio y los principios que gobiernan el dominio.
\item Los conceptos en las ontologías de dominio son usualmente especializaciones de conceptos ya definidos en la ontología de alto nivel. Lo mismo ocurre con las relaciones.
\end{itemize}

\subsubsection{Ontologías de Tareas}

\begin{itemize}
\item Describen el vocabulario relacionado a una tarea o actividad genérica, por ejemplo, de diagnóstico o de ventas, por medio de la especialización de los conceptos introducidos en las ontologías de alto nivel.
\item Proporcionan un vocabulario sistemático de los términos utilizados para resolver los problemas relacionados con las tareas que pueden o no pertenecer al mismo dominio.
\end{itemize}

\subsubsection{Ontologías de Aplicación}

\begin{itemize}
\item Describen conceptos que pertenecen a la vez a un dominio y a una tarea particular, por medio de la especialización de los conceptos de las ontologías de dominio y de tareas.
\item Generalmente corresponden a roles que juegan las entidades del dominio cuando ejecutan una actividad.
\item Contienen todas las definiciones necesarias para modelar el conocimiento de una determinada aplicación.
\item Ontologías de aplicación a menudo se extienden y se especializan en el vocabulario del dominio y de las ontologías de tareas para una aplicación dada.
\end{itemize}

\section{Arquitectura}

\subsection{Principios}

El desarrollo de la Web Semántica tiene lugar por pasos. Cada paso construye una capa encima de otra. Se logra consenso en pequeños pasos mas fácilmente.

\subsubsection{Compatibilidad Descendente}

Agentes con el conocimiento propio de una capa deberían también interpretar y usar información escrita en niveles inferiores (máximo provecho).

\subsubsection{Compresión Parcial Hacia Arriba}

Agentes con el conocimiento de una capa deberían ser capaces de tomar ventaja de información parcial de niveles superiores.

\subsection{Capas}

\subsubsection{Capa URI}

\begin{itemize}
\item Identificadores de recursos únicos, sin posibilidad de ambigüedad.
\item Puede ser una localización (URL), un nombre (URN) o ambos.
\end{itemize}

\subsubsection{Capa XML}

\begin{itemize}
\item Un lenguaje de etiquetas debe especificar:
	\begin{itemize}
	\item Las etiquetas permitidas.
	\item Las etiquetas requeridas.
	\item Cómo se distinguen las etiquetas del texto.
	\item Qué significan las etiquetas.
	\end{itemize}
\item XML sólo especifica las tres primeras, la cuarta es especificada por DTD.
\item Metalenguaje de etiquetas extensibles, se puede \textit{acomodar} a las necesidades de cada uno.
\item Esquemas que definen y restringen su estructura.
\item Base sintáctica.
\item Lenguaje que permite escribir documentos Web estructurados.
\item Utiliza un vocabulario definido por el usuario.
\item Establece relaciones básicas pero no una semántica.
\item Es más fácilmente accesible para máquinas:
	\begin{itemize}
	\item  Se describe cada elemento de información.
	\item Se definen las relaciones a través de la estructura anidada. 
	\end{itemize}
\item Es un lenguaje de marcas tal como HTML.
\item Fue diseñado para describir datos.
\item Las etiquetas no están predefinidos.
\item Usa un Document Type Definition (DTD) o un XML Schema para describir los datos.
\item XML con un DTD o XML Schema fue diseñado para ser auto-descriptivo.
\item Es recomendación de la W3C.
\item Es un metalenguaje que no tiene un conjunto fijo de etiquetas pero permite al usuario definir sus propias etiquetas.
\item Fue diseñado para describir datos.
\item Reduce la complejidad de la interpretación de los datos.
\item Mayor facilidad para expandir y actualizar un sistema.
\end{itemize}

\subsubsection{Capa RDF}

\begin{itemize}
\item Infraestructura para la Descripción de Recursos.
\item Modelo de datos básico.
\item Permite escribir sentencias simples acerca de recursos Web.
\item No depende de XML pero tiene una sintaxis basada en XML.
\item Recomendacion del W3C.
\item Estandariza la definición y uso de metadatos (útil para la representación de datos).
\item Usa la sintaxis de XML.
\item Soluciona las carencias de XML, y agrega semántica.
\item Lenguaje centrado en propiedades, no en recursos.
\item Posee semántica formal.
\item \textbf{Tripletas:}
	\begin{itemize}
	\item Cada tripleta representa una declaración de una relación entre los elementos 	denotados por los vínculos.
	\item Cada tripleta tiene 3 partes:
		\begin{itemize}
		\item Un sujeto.
		\item Un objeto.
		\item Un predicado (también llamado propiedad) que denota una relación. 
	\end{itemize}
	\item La dirección del arco es significativa: siempre apunta hacia el objeto.
	\item Conjunto tripletas: Grafo.
	\end{itemize}
\item \textbf{Recursos:}
	\begin{itemize}
	\item Podemos ver una \textit{cosa} o \textit{recurso} sobre lo que queremos hacer 	referencia.
	\item Cada recurso tiene un URI (Universal Resource Identifier).
	\item Un URI puede ser:
		\begin{itemize}
		\item Una URL (dirección web).
		\item Otra clase de identificador único.
		\end{itemize}
	\item Para nosotros una URI es el identificador de un recurso web.
	\end{itemize}
\item \textbf{Propiedades:}
	\begin{itemize}
	\item Son una clase especial de recursos.
	\item Describen relaciones entre recursos.
	\item También se identifican por URI's.
	\item Brinda un esquema único y global para nombrar a las cosas.
	\item Reduce el problema del manejo de homónimos de la representación distribuida.
	\item Homónimos son aquellos términos o palabras que, aunque se escriben o 				pronuncian de manera similar, tienen diferente valor gramatical, como por ejemplo: 	más y mas.
	\end{itemize}
\item \textbf{Sentencias:}
	\begin{itemize}
	\item Las sentencias establecen las propiedades de los recursos.
	\item Una sentencia es una tripleta del tipo objeto-atributo-valor.
	\item Consiste en un recurso, una propiedad y un valor.
	\item Los valores pueden ser recursos o literales (valores atómicos, strings).
	\end{itemize}
\item \textbf{Beneficios:}
	\begin{itemize}
	\item RDF tiene suficiente poder expresivo, como base sobre la cual otras capas de 	la arquitectura de la Web Semántica se pueden construir.
	\item La Web Semántica no se va a programar en RDF pero si con herramientas que 		van a traducir en forma automática representaciones de más alto nivel en RDF.
	\item Con RDF la información se mapea sin ambigüedad a un modelo.
	\item Como RDF es un estándar, trabajar sobre RDF equivale a trabajar en HTM en 		los primeros tiempos de la Web.
	\end{itemize}
\item \textbf{Desventajas:}
	\begin{itemize}
	\item RDF permite la afirmación de sentencias simples que consisten en sujeto-			predicado-objeto.
	\item No describe lo que estos elementos significan sino que describen las				relaciones que existen entre ellos.
	\end{itemize}
\end{itemize}

\subsubsection{Capa RDFS}

\begin{itemize}
\item Provee primitivas de modelado para la organización de recursos en jerarquías: clases, propiedades, relaciones de subclases y subpropiedades, y restricciones de dominio y rango.
\item Está basado en RDF.
\item Lenguaje primitivo para la definición de ontolgías.
\item RDF + definición de un vocabulario.
\item RDFS no provee clases ni propiedades particulares de una aplicación, sino que otorga un framework para describir esas clases y propiedades.
\item Expresiones RDF SCHEMA son expresiones RDF válidas.
\item Introduce conceptos ontológicos simples:
	\begin{itemize}
	\item Introduce el concepto de clase.
	\item Define cómo los recursos pueden describirse como pertenecientes a una o más 		clases.
	\item Describe jerarquía de clases y propiedades.
	\item Define dominio y rango de propiedades.
	\end{itemize}
\item Las clases en un RDF Schema son comparables a las clases en lenguajes de  programación orientada a objetos.
\item Los recursos pueden ser definidos como instancias de clases o subclases de clases.
\item Una ontología en RDFS debe comenzar con un nodo raíz RDF donde se incluyen los namespaces para las ontologías RDF y RDFS (ontologías de representación del conocimiento).
\item El uso de los namespaces permite utilizar los prefijos rdf y rdfs para las primitivas que pertenecen a RDF y RDFS.
\item \textbf{Clase:}
	\begin{itemize}
	\item Una clase representa una colección de recursos.
	\item Son recursos en si mismas identificados por URI's.
	\item Un recurso comienza a ser un miembro de una clase utilizando la propiedad 		rdf:type.
	\item Conceptos son clases y subclases en RDFS.
	\item Se referencian por nombre o URL a un recurso web.
	\item \textit{rdfs:subClassOf} indica que una clase es subclase de otra.
	\end{itemize}
\item \textbf{Propiedades:}
	\begin{itemize}
	\item Propiedades RDF son recursos.
	\item rdf:Property es la clase de todas las propiedades.
	\item Atributos de instancia de clases se definen como propiedades en RDFS.
	\item El dominio de estas propiedades es la clase a la que pertenece el atributo y 	el rango es el tipo del valor del atributo.
	\item No se definen restricciones de cardinalidad ni valores por omisión.
	\item Atributos de clase se representan de manera similar.
	\item El dominio de la propiedad se define como rdfs:class, y se incluye el valor		de la propiedad en la definición de la clase.
	\end{itemize}
\item Una referencia URI o un literal utilizado como nodo, identifica lo que el nodo representa.
\item Una referencia URI utilizada como predicado identifica una relación entre los elementos representados por los nodos que conecta.
\item Una referencia URI predicado puede también ser un nodo en el grafo.
\end{itemize}

\subsubsection{Capa OWL}

\begin{itemize}
\item Necesidad de lenguajes de ontologías mas potentes que expandan RDF-S y que permitan la representación de relaciones mas complejas entre recursos web.
\item Lenguaje para definir ontologías.
\item Estándar web.
\item Construido sobre RDF para procesar información en la web.
\item Diseñado para ser interpretado por computadoras, no para ser leído por las personas.
\item Utiliza sintaxis XML.
\item OWL es similar a RDF pero:
	\begin{itemize}
	\item Lenguaje más potente.
	\item Provee mayor interoperabilidad.
	\item Mayor vocabulario.
	\item Mejor sintaxis.
	\end{itemize}
\item Limitaciones de RDFS:
	\begin{itemize}
	\item  Expresar la disyunción de clases.
	\item Definir clases como combinación de otras (unión, intersección o 					complemento).
	\item Expresar restricciones sobre la cardinalidad de propiedades.
	\item Describir propiedades específicas de las propiedades.
	\end{itemize}
\item Requerimientos para lenguajes de representación de ontologías (extensión de 	RDFS):
	\begin{itemize}
	\item Una sintaxis bien definida: condición necesaria para información procesable 		por máquinas.
	\item Una semántica formal: prerequisito para soporte de razonamiento.
	\item Soporte de razonamiento: verificar la consistencia de la ontología.
	\item Suficiente poder expresivo.
	\end{itemize}
\item La información Web tiene un significado preciso.
\item La información Web puede ser procesada por computadoras.
\item Las computadoras pueden integrar la información de la web.
\item OWL está diseñado para:
	\begin{itemize}
	\item Proveer una forma común para procesar el contenido de la web en vez de  			mostrarlo.
	\item Permitir la lectura por aplicaciones en vez de humanos.
	\end{itemize}
\item En una ontología OWL encontramos:
	\begin{itemize}
	\item Clases + jerarquía de clases.
	\item Propiedades (Slots) / values.
	\item Relaciones. Relaciones entre clases (herencia, disyunción, equivalencia).
	\item Restricciones. Restricciones sobre las propiedades (tipo, cardinalidad).
	\item Características de propiedades (transitividad,...).
	\item Anotaciones.
	\item Individuos.
	\end{itemize}
\item Tareas de razonamiento: clasificación, chequeo de consistencia.
\end{itemize}

\subsubsection{Capa Lógica}

\begin{itemize}
\item Enfatizar lenguajes ontológicos.
\item Desarrollo de aplicaciones específicas de conocimiento declarativo.
\end{itemize}

\subsubsection{Capa de Prueba}

\begin{itemize}
\item Generación de prueba, validación.
\end{itemize}

\subsubsection{Capa de Confianza}

\begin{itemize}
\item Firma digital.
\item Recomendaciones, certificaciones.
\item Seguridad y calidad en operaciones e información.
\end{itemize}

\subsection{Componentes para una Web Semántica:}

\begin{itemize}
\item XML nos da la sintaxis para documentos estructurados, pero no agrega semántica.  Permite estructurar documentos según vocabularios definidos por el usuario.
\item XML Schema restringe la estructura de documentos XML y extiende a XML con datatypes.
\item RDF es un modelo de datos para objetos ("recursos") y relaciones entre ellos. Provee semántica simple para este modelo de datos, y puede ser representado con sintaxis de XML. Proporciona un modelo para describir aserciones sobre recursos Web.
\item RDF Schema es un vocabulario para describir clases y propiedades de recursos RDF, usando semántica para jerarquías generalizadas de esas propiedades y clases.   Proporciona primitivas para organizar objetos en jerarquías (ontologías simples).
\item OWL agrega vocabulario para describir propiedades y clases: entre otros, relaciones entre clases, cardinalidad, igualdad, características de propiedades, etc. Permite expresar relaciones más complejas entre objetos (ontologías complejas).
\end{itemize}

\end{document}